\documentclass{beamer}
\usetheme{Darmstadt}
\usepackage{bibentry}
\usepackage{natbib}
% \usepackage[backend=biber]{biblatex}
% \addbibresource{refs.bib}

%Information to be included in the title page:
\title{Fundamental Structure of Statistical Papers}
\author{Shuiyi Hu}
\institute{University of Connecticut}
\date{October 2023}
\setbeamertemplate{footline}[frame number]


\begin{document}

\frame{\titlepage}


\begin{frame}{Fundamental Structure of Statistical Papers}
    \begin{columns}
        \column{0.5\textwidth}
            \centering
            \begin{itemize}
                \item Title
                \item Abstract
                \item Keywords
                \item Introduction
                \item Data
                \item Methods
            \end{itemize}
        \column{0.5\textwidth}
            \centering
            \begin{itemize}
                \item Simulation
                \item Application
                \item Discussion and Conclusion
                \item Appendix
                \item Acknowledgements
                \item Reference
            \end{itemize}
        \end{columns}
    \end{frame}
    
    
    \begin{frame}{Title}
    The title of a paper is usually determined when the paper is close to completion.\\
    Goals:
    \begin{itemize}
        \item Predict the content of the research paper
        \item Be interesting to the reader
        \item Reflect the tone of the writing
        \item Contain important keywords
    \end{itemize}
    
    \end{frame}
    
    
    \begin{frame}{Title}
    
    Tips \citep{Yan2023}:
    \begin{itemize}
        \item Be informative by including these aspects: topic, method(s), data, and results.
        \item Consider adding a subtitle to give more specifics about the paper.
        \item Use appropriate critical keywords to increase the discoverability of the paper.
        \item Follow the requirements from the instructions or journals.
        \item Keep it as concise as possible.
    \end{itemize}
    
    \end{frame}
    
    
    \begin{frame}{Abstract}
    \begin{itemize}
        \item A balance between the twin goals of brevity and maximal information content should be carefully sought.
        \item Make sure each high point is included.
        \item Recommendations for length must be more case dependent.
        \item Mathematical notation is rarely useful in the abstract.
    \end{itemize}
    \end{frame}
    
    
    \begin{frame}{Abstract}
    \begin{itemize}
        \item Open with a sentence to establish the importance of the subject of the paper.
        \item Identify a gap in the literature to set up the background of the paper.
        \item Highlight the novelty/contributions of the paper.
        \item It must make sense when read in isolation for those who only read the abstract, and must also provide a clear and accurate summary of the manuscript for readers who read the entire manuscript.
        \item Should not include citations.
    \end{itemize}
    \end{frame}
    
    
    \begin{frame}{Keywords}
    \begin{itemize}
        \item Keywords are words in addition to those in the title that attract search queries.
        \item Including the most relevant keywords helps other researchers find your paper.
        \item Tips:
        \begin{itemize}
            \item No need to repeat anything in the title already.
            \item List them in alphabetical order.
            \item Contain words and phrases that suggest what the topic is about.
        \end{itemize}
    \end{itemize}
    \end{frame}
    
    
    \begin{frame}{Introduction}
    \begin{itemize}
        \item The introduction section is always the first section of a paper.
        \item The purpose of the introduction is to stimulate the reader’s interest and to provide background information which is pertinent to the study.
        \item The introduction section guides the readers from a general subject area to the narrow topic of the paper.
    \end{itemize}
    \end{frame}
    
    
    \begin{frame}{Introduction}
    \begin{itemize}
        \item The introduction sections need to explain the importance of the topic of the paper, provide the background of the research work, and highlight the contributions of the work.
        \item The introduction is typically outlined at the very beginning of the writing process, but completed towards the end after the other sections have been written.
    \end{itemize}
    \end{frame}
    
    
    \begin{frame}{Introduction}
    An introduction often contains the following items:
    \begin{itemize}
        \item An overview of the topic.
        \item Existing works.
        \item A gap.
        \item Contributions.
        \item A roadmap.
    \end{itemize}
    \end{frame}
    
    
    \begin{frame}{Data}
    \begin{itemize}
        \item Who collected the data?
        \item How was the data collected? Sampling frame? Sampling approach?
        \item What period or range does the data cover?
        \item Why does the data help answer the research question?
        \item What exploratory analyses are done?
    \end{itemize}
    \end{frame}
    
    
    \begin{frame}{Methods}
    \begin{itemize}
        \item Establish notation.
        \item Notation is much easier to digest if the reader first understands the main idea at an intuitive level.
    \end{itemize}
    \end{frame}
    
    
    \begin{frame}{Methods}
    \begin{itemize}
        \item What are the observed data?
        \item What are the models?
        \item What are the parameters to be estimated?
        \item How are the point estimators obtained?
        \item How are the uncertainty (standard errors) of the point estimators assessed?
        \item How are the variances of the point estimators estimated?
        \item How are the null distribution of the testing statistics established?
        \item Clearly state the assumptions and claims of theoretical results.
    \end{itemize}
    \end{frame}
    
    
    \begin{frame}{Simulation}
    ADEMP \citep{Morris2019}:
    \begin{itemize}
        \item Aims
        \item Data generating mechanism
        \item Estimand/target of analysis
        \item Methods
        \item Performance measures
    \end{itemize}
    \end{frame}
    
    
    \begin{frame}{Simulation}
    \begin{itemize}
        \item Presentation of results
        \begin{itemize}
            \item May be included in a Results section.
            \item For each table/figure, write down the bullet points to convey to the readers.
            \item Group the bullet points in blocks and put the blocks in a logical order.
            \item Within each block, put the bullet points in the right logical order.
            \item Some (shorter) blocks can be converted into proper paragraphs in the final paper, while other (longer) blocks may remain in bullet form.
        \end{itemize}
    \end{itemize}
    \end{frame}
    
    
    \begin{frame}{Application}
    \begin{itemize}
        \item Report the statistical analyses in tables/figures.
        \item When summarizing from tables/figures, paint the big picture, rather than reiterating all of the little details.
        \item Discussions to link the analyses back to the substantive topic.
    \end{itemize}
    \end{frame}
    
    
    \begin{frame}{Discussion and Conclusion}
    \begin{itemize}
        \item A summary of the contributions of the research.
        \item The research question posed as the `need’ of the introduction must be answered here.
        \item Limitations of the current study.
        \item Future directions.
    \end{itemize}
    \end{frame}
    
    
    \begin{frame}{Appendix}
    \begin{itemize}
        \item Technical details that would otherwise break the flow of the main text.
        \begin{itemize}
            \item proofs
            \item algorithms
        \end{itemize}
        \item Data source details.
    \end{itemize}
    \end{frame}
    
    
    \begin{frame}{Acknowledgements}
    \begin{itemize}
        \item Optional
        \item Be used to acknowledge certain individuals who have contributed to the research and/or success of the manuscript
        \item funding agency and funding mechanism are typically included here unless otherwise specified
    \end{itemize}
    \end{frame}
    
    
    \begin{frame}{Reference}
    \begin{itemize}
        \item Every reference cited in the paper should appear here.
        \item References not cited should not appear here.
        \item All are automatically taken care of by BibTeX.
        \item Styles are controlled by bib style (.bst file).
    \end{itemize}
    \end{frame}
    
    
    \begin{frame}{Example}
    Example \citep{Li2021}
    \begin{itemize}
        \item Title: Uncertainty in optimal fingerprinting is underestimated
        \item Keywords: bootstrap calibration, climate change, confidence interval, detection and attribution, natural climate variability,
    uncertainty quantification
        \item Abstract
        \item Introduction
        \item Methods
        \begin{itemize}
            \item Optimal fingerprinting
            \item Simulation experiment design
            \item Data preparation for application to mean
    temperature
        \end{itemize}
    \end{itemize} 
    \end{frame}
    
    
    \begin{frame}{Example(Continued)}
    \begin{itemize}
        \item Results
        \begin{itemize}
            \item Simulated experiments (with figures)
            \item Detection and attribution of changes in mean temperature
    temperature (with figures)
        \end{itemize}
        \item Conclusion
        \item Data availability statement
        \item Funding, Author contributions, Conflict of interest
        \item Appendix
        \item ORCID iDs, References
    \end{itemize} 
    \end{frame}


\begin{frame}
  \frametitle{References}
  \bibliography{refs}
  \bibliographystyle{mcap}
\end{frame}


\end{document}